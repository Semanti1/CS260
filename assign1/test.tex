\documentclass{article}
\usepackage{amsmath}
\usepackage{listings}
\usepackage{algorithm2e}
\begin{document}
\title {CS 260 Homework 1}
\author{Semanti Basu}
\maketitle
\section{First Section-A section header}
Numbered Section 
\subsection{Subsection-A subsection header}
I am a subsection

\section*{Example of bulleted list}


\begin{itemize}
\item ice-cream
\item soda
\item veggies
\item pasta
\item bread
\end{itemize}



\section*{Example of numbered list.}




\begin{enumerate}
\item Data-Structures
\item Web and Mobile app development
\item Systems Programming
\item Design with microcontrollers
\item Transforms
\item Engr 202
\end{enumerate}



\section*{Two paragraphs about me.}




Hello, this is Simmy. I like robotics. I have done projects on autonomous drones. I have also tried using Artificial Intelligence on robots. I am part of Drexel Robotics club.

My hobby is reading. I love reading books. I have had different favourites at different times. Little women,Famous five, Harry Potter, The Diary of a Young Girl, Angels and Demons,Pride and Prejudic etc

\section*{Bold and Italics}
\begin{itemize}
\item \textbf {Bold font}
\item \textit {Italic font}
\end{itemize}

\section*{Here is a math equation in a sentence:}




\begin{enumerate}
\item Example of doing integrals $\int e^{i k \theta} d \theta$
\end{enumerate} 


\section*{Here is a math equation on a line:}


\begin{equation}
\sum_{k=1}^{n} k= \frac {n (n+1)}{2}
\end{equation}



\section*{Example of align.}



\begin{align}
2x+3y=10\\
2x-3y=6\\
4x=16\\
x=\frac {16}{4}\\
x=4\\
6y=4\\
y=\frac {4}{6}
\end{align}


\section*{An inline code block.}


\begin{lstlisting}
int sum(int a,int b)
{
	return a+b;
}
\end{lstlisting}


\section*{Here is a pseudocode .}


\begin{algorithm}[H]
\KwData{integer a and array A as input}
\KwResult{Find out if a is present in array A}
\For{i=0,i $\le$ length(a)-1,i++}{
\If{A[i]==a}{

 return True\;

}
}
return false\;
\caption{Searching for a given element in a given array and returning if the element exixts in array or not}
\end{algorithm}
\end{document}
