\documentclass{article}

\usepackage{amsmath}
\usepackage{listings}
\usepackage{algorithm2e}
\usepackage{enumitem}
\usepackage{graphicx}
\graphicspath{./}
\begin{document}
\title {CS 260 Homework 8}
\author{Semanti Basu}
\maketitle


\section*{Question 1}
\subsection*{part a}
h(1)=1 mod 7=1\\
h(8)=8 mod 7=1\\
h(27)=27 mod 7=6\\
h(64)=64 mod 7=1\\
h(125)=125 mod 7=6\\
h(216)=216 mod 7=6\\
h(343)=343 mod 7=0\\

Open Hash Table



\begin{center}
\begin{tabular}{c|c|}
\hline
Row & List\\ \hline
0$\rightarrow$ & [343]\\ \hline
1$\rightarrow$ & [1,8,64]\\ \hline
2$\rightarrow$ & []\\ \hline
3$\rightarrow$ & []\\ \hline
4$\rightarrow$ & []\\ \hline
5$\rightarrow$ & []\\ \hline
6$\rightarrow$ & [27,125,216]\\ \hline
\end{tabular}
\end{center}


\subsection*{part b}


Closed Hash Table

1=slot 1\\
8=slot(1+1)=slot 2\\
27=slot 6\\
64=slot(1+1+1)=slot 3\\
125=slot(6+1)=slot 0\\
216=slot(6+1+1+1+1+1)=slot 4\\
343=slot(0+1+1+1+1+1)=slot 5\\

\begin{center}
\begin{tabular}{c|c|}
\hline
Slot & Number\\ \hline
0$\rightarrow$ & 125\\ \hline
1$\rightarrow$ & 1\\ \hline
2$\rightarrow$ & 8\\ \hline
3$\rightarrow$ & 64\\ \hline
4$\rightarrow$ & 216\\ \hline
5$\rightarrow$ & 343\\ \hline
6$\rightarrow$ & 27\\ \hline
\end{tabular}
\end{center}



\section*{Question 2}


h(23)=23 mod 5=3\\
h(48)=48 mod 5=3\\
h(35)=35 mod 5=0\\
h(4)=4 mod 5=4\\
h(10)=10 mod 5=0\\

23=slot 3\\
48=slot (3+1)=slot 4\\
35=slot 0\\
4=slot(4+1+1)=slot 1\\
10=slot(0+1+1)=slot 2\\

\begin{center}
\begin{tabular}{c|c|}
\hline
Slot & Number\\ \hline
0$\rightarrow$ & 35\\ \hline
1$\rightarrow$ & 4\\ \hline
2$\rightarrow$ & 10\\ \hline
3$\rightarrow$ & 23\\ \hline
4$\rightarrow$ & 48\\ \hline
\end{tabular}
\end{center}



\section*{Question 3}
\subsection*{part a}

This is not a good hash function because there are several strings which have same length. So there will be multiple collisions which will have to be dealt with.

For Ex:

h1("tea")=h1("you")=h1("cat")=h1("mat")=h1("rat")=h1("dog")=h1("hat")=3\\
h1("four")=h1("roar")=h1("soar")=h1("sear")=h1("sore")=4\\


There are a large number of collisions in each case which is not practical from a coding perspective.

\subsection*{part b}

This is not a good hash function because it will return a different hash for the same string. It is not guaranteed to return the same hash value. So it will be impossible to look 
up anything in the hash table.

For ex:

h1("hello")=3 but h1("hello")=6 when called a second time. The function can have different values for same string, hence it is not a good hash function.
\end{document}


