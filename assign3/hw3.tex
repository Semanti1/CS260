\documentclass{article}
\usepackage{amsmath}
\usepackage{listings}
\usepackage{algorithm2e}
\usepackage{enumitem}
\begin{document}
\title {CS 260 Homework 3}
\author{Semanti Basu}
\maketitle


\section*{Question 1}

Bubble sort
\begin{itemize}
\item (1,7,3,2,0,5,0,8)
\item (1,3,7,2,0,5,0,8)
\item (1,3,2,7,0,5,0,8)
\item (1,3,2,0,7,5,0,8)
\item (1,3,2,0,5,7,0,8)
\item (1,3,2,0,5,0,7,8)

\item (1,3,2,0,5,0,7,8) first iteration
\item (1,2,3,0,5,0,7,8)
\item (1,2,0,3,5,0,7,8)
\item (1,2,0,3,5,0,7,8)
\item (1,2,0,3,0,5,7,8)

\item (1,2,0,3,0,5,7,8)second iteration

\item (1,0,2,3,0,5,7,8)
\item (1,0,2,0,3,5,7,8)

\item (1,0,2,0,3,5,7,8)third iteration

\item (0,1,2,0,3,5,7,8)
\item (0,1,0,2,3,5,7,8)

\item (0,1,0,2,3,5,7,8)fourth iteration

\item (0,0,1,2,3,5,7,8)

\item (0,0,1,2,3,5,7,8) fifth iteration (sorted)
\end{itemize}

Selection sort
\begin{itemize}
\item (1,7,3,2,0,5,0,8)
\item (1,7,3,2,0,5,0,8) first iteration
\item (1,3,7,2,0,5,0,8) second iteration
\item (1,3,2,7,0,5,0,8)
\item (1,2,3,7,0,5,0,8) third
\item (1,2,3,0,7,5,0,8)
\item (1,2,0,3,7,5,0,8)
\item (1,0,2,3,7,5,0,8)
\item (0,1,2,3,7,5,0,8)fourth
\item (0,1,2,3,5,7,0,8)fifth
\item (0,1,2,3,5,0,7,8)
\item (0,1,2,3,0,5,7,8)
\item (0,1,2,0,3,5,7,8)
\item (0,1,0,2,3,5,7,8)
\item (0,0,1,2,3,5,7,8)(sorted)
\end{itemize}





\section*{Question 2}
Initial array (22, 36, 6, 79, 26, 45, 75, 13, 31, 62, 27, 76, 33, 16, 62, 47) 47 is pivot
\begin{itemize}
\item (22, 36, 6, 26, 45, 13, 31, 27, 33, 16, 47, 76, 75, 62, 62, 79)
\item (6, 13, 16, 26, 45, 36, 31, 27, 33, 22, 47, 76, 75, 62, 62, 79)
\item (6, 13, 16, 26, 45, 36, 31, 27, 33, 22, 47, 76, 75, 62, 62, 79)
\item (6, 13, 16, 22, 45, 36, 31, 27, 33, 26, 47, 76, 75, 62, 62, 79)
\item (6, 13, 16, 22, 26, 36, 31, 27, 33, 45, 47, 76, 75, 62, 62, 79)
\item (6, 13, 16, 22, 26, 36, 31, 27, 33, 45, 47, 76, 75, 62, 62, 79)
\item (6, 13, 16, 22, 26, 31, 27, 33, 36, 45, 47, 76, 75, 62, 62, 79)
\item (6, 13, 16, 22, 26, 27, 31, 33, 36, 45, 47, 76, 75, 62, 62, 79)
\item (6, 13, 16, 22, 26, 27, 31, 33, 36, 45, 47, 76, 75, 62, 62, 79)
\item (6, 13, 16, 22, 26, 27, 31, 33, 36, 45, 47, 62, 62, 76, 75, 79)
\item (6, 13, 16, 22, 26, 27, 31, 33, 36, 45, 47, 62, 62, 75, 76, 79)
\end{itemize}


\section*{Question3}

\subsection*{Part a}

\begin{align}
T(n)=4T(\frac{n}{3}) + n\\
    =4[4T(\frac{n}{3}) + \frac{n}{3}] + n\\
    =16T(\frac{n}{9})+ 4\frac{n}{3} + n\\
    =64T(\frac{n}{27})+ 16\frac{n}{9}+4\frac{n}{3} + n\\
    =4^kT(\frac{n}{3^k}) + 3((\frac{4}{3})^k-1)n\\
\frac{n}{3^k}=1\\
k=\log_3 n\\
Plugging in:
T(n)=4^{\log_3 n}T(1)+3n((\frac{4}{3})^{\log_3 n-1})\\
    =O(n^{\log_3 4})\\
\end{align}


\subsection*{Part b}

\begin{align}
T(n)=4T(\frac{n}{3}) + n^2\\
    =4[4T(\frac{n}{3}) + \frac{n^2}{9}] + n^2\\
    =16T(\frac{n}{9})+ 4\frac{n^2}{9} + n^2\\
    =64T(\frac{n}{27})+ 16\frac{n^2}{81}+4\frac{n^2}{9} + n^2\\
    =4^kT(\frac{n}{3^k}) + \frac{9}{5}((1-\frac{4}{9}^k))n^2\\
\frac{n}{3^k}=1\\
k=\log_3 n\\
Plugging in:
T(n)=4^{\log_3 n}T(1)+\frac{9}{5}((1-\frac{4}{9}^{\log_3 n}))n^2\\
    =O(n^2)\\
\end{align}


\subsection*{Part c}


\begin{align}
T(n)=9T(\frac{n}{3}) + n^2\\
    =9[9T(\frac{n}{9}) + \frac{n^2}{9}] + n^2\\
    =81T(\frac{n}{9})+ 2n^2\\
    =81[9T(\frac{n}{27})+\frac{n^2}{81}]+2n^2\\
    =9^kT(\frac{n}{3^k}) + kn^2\\
\frac{n}{3^k}=1\\
k=\log_3 n\\
Plugging in:
T(n)=9^{\log_3 n}T(1)+{\log_3 n}n^2\\
    =n^2 + n^2{\log_3 n}\\
    =O(n^2{\log_3 n})\\
\end{align}

\section*{Question4}

\subsection*{Part a}

 $T(n)=T(n/2)+1$
Using Master's theorem:
\begin{align}
a=1\\
b=2\\
c=\log_2 1\\
 =0\\
f(n)=n^0\\
    =n^c\\
f(n)=O(n^c {\log_2 n}^k)\\
k=0\\
T(n)=O(n^c {\log_2 n}^{k+1})\\
    =O(\log_2 n)\\
T(n)=\Omega(\log_2 n)\\
\end{align}

\subsection*{Part b}

 $T(n)=2T(n/2)+\log n$
Using Master's theorem:
\begin{align}
a=2\\
b=2\\
c=\log_2 2\\
 =1\\
f(n)=\log n\\
n^c=n
n^c>f(n) \text{n grows faster than log n}\\
\text{By case 1 of Master theorem}\\
f(n)=O(n^{c-\epsilon})\\
T(n)=O(n)\\
T(n)=\Omega(n)\\
\end{align}

\subsection*{Part c}

$T(n)=2T(n/2)+ n$
Using Master's theorem:
\begin{align}
a=2\\
b=2\\
c=\log_2 2\\
 =1\\
f(n)=n\\
=n^c\\
\text{By case 2 of Master's theorem}\\
f(n)=O(n^c {\log_2 n}^k)\\
k=0\\
T(n)=O(n^c {\log_2 n}^{k+1})\\
    =O(n\log_2 n)\\
T(n)=\Omega(n\log_2 n)\\
\end{align}


\subsection*{Part d}

$T(n)=2T(n/2)+ n^2$
Using Master's theorem:
\begin{align}
a=2\\
b=2\\
c=\log_2 2\\
 =1\\
f(n)=n^2\\
\text{By case 3 of master theorem}\\
n^c<f(n)
f(n)=\Omega(n^{c+\epsilon})
T(n)=O(f(n))\\
    =O(n^2)\\
T(n)=\Omega(f(n))\\
    =\Omega(n^2)\\
\end{align}











\end{document}















